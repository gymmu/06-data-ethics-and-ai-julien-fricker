\documentclass{article}

\usepackage[ngerman]{babel}
\usepackage[utf8]{inputenc}
\usepackage[T1]{fontenc}
\usepackage{hyperref}
\usepackage{csquotes}

\usepackage[
    backend=biber,
    style=apa,
    sortlocale=de_DE,
    natbib=true,
    url=false,
    doi=false,
    sortcites=true,
    sorting=nyt,
    isbn=false,
    hyperref=true,
    backref=false,
    giveninits=false,
    eprint=false]{biblatex}
\addbibresource{../references/bibliography.bib}

\title{Notizen zum Projekt Data Ethics}
\author{Julien Fricker}
\date{\today}

\begin{document}
\maketitle
\title{Stichworte zum Projekt: KI, Daten und Ethik}

\section{Vorteile und Chancen:}

Effizienzsteigerung in verschiedenen Branchen (Produktion, Lieferketten)
Beschleunigung von wissenschaftlichen Entdeckungen (Mustererkennung, Modellierung)
Verbesserungen im Gesundheitswesen (Diagnosen, Therapien)

\section{Ethische Herausforderungen:}

Datenschutz und Privatsphäre (Missbrauch sensibler Daten)
Diskriminierung und Bias (Vorurteile in Algorithmen)
Verantwortung und Haftung (Unklarheit im Schadensfall)

\section{Fazit:}

KI und Big Data bieten enorme Vorteile, aber auch ethische Herausforderungen.
Datenschutz, Fairness und Verantwortlichkeit müssen zentral sein.
Ethische Leitlinien und Datenschutzmassnahmen sind notwendig.
Entwickler, Regulierungsbehörden und Gesellschaft müssen zusammenarbeiten.




\end{document}
