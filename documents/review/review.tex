\documentclass{article}

\usepackage[ngerman]{babel}
\usepackage[utf8]{inputenc}
\usepackage[T1]{fontenc}
\usepackage{hyperref}
\usepackage{csquotes}

\usepackage[
    backend=biber,
    style=apa,
    sortlocale=de_DE,
    natbib=true,
    url=false,
    doi=false,
    sortcites=true,
    sorting=nyt,
    isbn=false,
    hyperref=true,
    backref=false,
    giveninits=false,
    eprint=false]{biblatex}
\addbibresource{../references/bibliography.bib}

\title{Review des Papers "Ethik im Umgang mit Daten" von Linus Meyer}
\author{Julien Fricker}
\date{\today}

\begin{document}
\maketitle



\maketitle

\section*{Korrektheit der Informationen und Übereinstimmung mit dem Text}
Der Text bietet eine fundierte Einführung in die Künstliche Intelligenz (KI), ihre Anwendungen und ethischen Herausforderungen. Die dargestellten Informationen sind präzise und decken die wesentlichen Aspekte der KI ab. Die Definition und die Beispiele für Anwendungen in den Bereichen Medizin, Wirtschaft und Alltag sind passend und aktuell.

\section*{Korrektheit der Grammatik}
Die Grammatik ist größtenteils korrekt, jedoch gibt es einige kleinere Fehler:
\begin{itemize}
    \item \textbf{"grossen Einfluss"} sollte zu \textbf{"großen Einfluss"} geändert werden.
    \item \textbf{"KI-Automatisierung kann auch zu Arbeitsplatzverlusten führen"} könnte klarer formuliert werden als \textbf{"Die Automatisierung durch KI kann auch zu Arbeitsplatzverlusten führen"}.
    \item \textbf{"was kritisch in Bezug des Datenschutzes und der Privatsphäre ist"} sollte zu \textbf{"was kritisch in Bezug auf Datenschutz und Privatsphäre ist"} geändert werden.
\end{itemize}

\section*{Positive Kritik}
\begin{itemize}
    \item \textbf{Struktur und Klarheit}: Der Text ist klar strukturiert und gut gegliedert, was das Verständnis erleichtert.
    \item \textbf{Beispiele und Anwendungen}: Die Auswahl der Beispiele ist gelungen und veranschaulicht die breite Palette der KI-Einsatzmöglichkeiten.
    \item \textbf{Ethische Überlegungen}: Die Erwähnung ethischer Fragen und Hinweise auf Initiativen wie die UNESCO und das Europäische Parlament zeigen, dass der Autor die Bedeutung dieser Aspekte erkannt hat.
\end{itemize}

\section*{Negative Kritik}
\begin{itemize}
    \item \textbf{Vertiefung der ethischen Diskussion}: Die ethischen Aspekte könnten ausführlicher behandelt werden. Konkrete Beispiele für ethische Dilemmata und mögliche Lösungsansätze würden die Diskussion bereichern.
    \item \textbf{Ausgewogenheit der Vor- und Nachteile}: Während die positiven Aspekte detailliert sind, könnten die negativen Aspekte weiter ausgearbeitet werden, z.B. durch spezifische Daten und Studien zu Arbeitsplatzverlusten.
    \item \textbf{Quellenverweise im Text}: Direkte Verweise auf die verwendeten Quellen im Text würden die Glaubwürdigkeit der Informationen erhöhen.
\end{itemize}

\section*{Übereinstimmung mit den Quellen}
Die angegebenen Quellen decken die behandelten Themen gut ab. Es wäre jedoch hilfreich, direkte Verweise auf diese Quellen im Text zu haben, um die Aussagen besser zu untermauern. Beispiele:
\begin{itemize}
    \item \textbf{Medizinische Anwendungen}: Informationen stimmen mit den auf \textit{soxes.ch} und \textit{ZDF} bereitgestellten Informationen überein.
    \item \textbf{Ethik der KI}: Ethikrichtlinien und Bedenken stimmen mit den Informationen von \textit{UNESCO} und \textit{IBM} überein.
    \item \textbf{Diskriminierung und Vorurteile}: Bedenken in Bezug auf Diskriminierung stimmen mit den Inhalten auf \textit{SRF} und \textit{HateAid} überein.
\end{itemize}

\section*{Fazit}
Der Text bietet eine gute Einführung in die Künstliche Intelligenz und behandelt relevante Themen auf verständliche Weise. Mit einigen grammatikalischen Verbesserungen und einer tiefergehenden Diskussion der negativen Aspekte und ethischen Herausforderungen könnte der Text weiter verbessert werden. Die Quellenangaben sind korrekt und relevant, sollten aber im Text selbst deutlicher referenziert werden.

\end{document}

\printbibliography

\end{document}
