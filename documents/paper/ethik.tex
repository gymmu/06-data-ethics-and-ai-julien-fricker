\chapter{Ethische Herausforderungen}
\label{chap:ethik}

\section{Datenschutz und Privatsphäre}

Da die KI Zugang zu so enormen Datenmengen hat, wirft dies Fragen zum Schutz der Privatsphäre auf. Es könnte dazu kommen, dass sensible Daten missbraucht werden, wenn sie nicht ausreichend geschützt sind. Deshalb ist es entscheidend, Datenschutzrichtlinien zu entwickeln und durchzusetzen, die den Missbrauch von persönlichen Daten verhindern. Außerdem sollten Daten nur mit Zustimmung der Betroffenen verwendet werden.

\section{Diskriminierung und Bias (Voreingenommenheit)}

Die meisten KI-Systeme lernen aus Daten, die menschliche Entscheidungen, Verhaltensweisen und Bewertungen widerspiegeln. Dies kann dazu führen, dass Vorurteile oder Diskriminierung verstärkt werden. Auch kann es zu unfairen Entscheidungen in Bereichen wie Personalwesen, Strafjustiz oder Finanzen führen. Deshalb ist es wichtig, die Algorithmen transparent zu gestalten und sie regelmäßig auf Bias zu prüfen. Generell sollte die Entwicklung von KI fair und gerecht sein und die Daten, auf denen ein System basiert, repräsentativ und ausgewogen sein.

\section{Verantwortung und Haftung}

Dieses Thema ist sehr komplex, nehmen wir als Beispiel einen Autounfall durch ein KI-gesteuertes Fahrzeug. Wer haftet für diesen Vorfall? Die Entwickler? Die Anwender? Die KI selbst? Dies sind Fragen, bei welchen Ethik eine große Rolle spielt. Bei einem obengenannten Fall hat man sich so entschieden, dass der Entwickler der KI und der Betreiber die Schuld tragen. 
