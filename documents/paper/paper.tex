\documentclass{report}

\usepackage[ngerman]{babel}
\usepackage[utf8]{inputenc}
\usepackage[T1]{fontenc}
\usepackage{hyperref}
\usepackage{csquotes}
\usepackage[a4paper]{geometry}

\usepackage[
    backend=biber,
    style=apa,
    sortlocale=de_DE,
    natbib=true,
    url=false,
    doi=false,
    sortcites=true,
    sorting=nyt,
    isbn=false,
    hyperref=true,
    backref=false,
    giveninits=false,
    eprint=false]{biblatex}
\addbibresource{../references/bibliography.bib}


\title{Ethik im Umgang mit Daten}
\author{Julien Fricker}
\date{\today}


\begin{document}

\maketitle

\tableofcontents

\chapter{Einleitung}

Die rasante Entwicklung der Künstlichen Intelligenz (KI) und der fortschreitende Umgang mit großen Datenmengen (Big Data) haben das Potenzial, unser Leben weitgehend zu verändern. Die KI kommt in zahlreichen Bereichen zum Einsatz, von einem Chatbot, der einem die Arbeit des Recherchierens leichter macht oder einem die Matheaufgaben erklären kann, bis zu medizinischen Diagnosen. 

Während die KI unseren Alltag zu vereinfachen scheint, kommen jedoch auch komplexe ethische Fragen auf. In diesem Projekt werde ich über die Beziehungen zwischen KI, Daten und Ethik sprechen, über Chancen und Schwierigkeiten informieren und manche ethische Fragen behandeln.

\input{chap_methode.tex}

\section{Etwas mit Quellen}

Etwas mit Änderung hier am Ende.

Wenn ich eine Quelle zitieren möchte, kann ich das ganze einfach am Ende des Satzes machen \citep{example}. Oder wie \citet{example} sagt, auch mitten im Text.

\printbibliography

\end{document}
