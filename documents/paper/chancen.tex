\chapter{Vorteile und Chancen in Bezug auf Datenverarbeitung}
\label{chap:chancen}

\section{Effizienzerhöhung in verschiedenen Branchen}

Die KI optimiert Produktionsprozesse durch Verringerung von Fehlern in der Industrie und verbessert Lieferketten, indem sie z.B. Vorhersagen über Nachfrage und Lieferzeiten macht. Dadurch steigt die Produktivität und es werden vor allem Kosten gespart. 

\section{Wissenschaft und Forschung}

Da die KI in der Lage ist, große Datenmengen zu analysieren, können wissenschaftliche Entdeckungen beschleunigt werden. Die KI kann durch die Musterung von Datensätzen eigenständig Lösungen für Problemstellungen finden bzw. neuartige Modelle erstellen. Die Geschwindigkeit oder Skalierbarkeit von Untersuchungsmethoden kann sich durch die KI erheblich verbessern. Wissenschaftlich können zum Beispiel KI-Übersetzungsprogramme nützlich sein, um Publikationen in Fremdsprachen zugänglich zu machen, was eine Menge Arbeit erspart und somit effizienter und billiger ist.

\section{Gesundheitswesen}

Auch in der Medizin sind die Fähigkeiten von KI-Programmen von großem Nutzen. Die KI kann durch die Analyse von medizinischen Daten Muster und Anomalien erkennen, die menschlichen Ärzten entgehen könnten. Dies führt zu präziseren Diagnosen und die Erforschung von Krankheiten wird durch Big Data auf einer neuen Ebene möglich. Dadurch wird die Entwicklung neuer Therapien beschleunigt. 
